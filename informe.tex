\documentclass[12pt, a4paper]{article}
\usepackage[utf8]{inputenc}
\usepackage[spanish]{babel}
\usepackage{graphicx}
\usepackage{amsmath}
\usepackage{hyperref}

\title{Nombre del Proyecto}
\author{Autor(es)}
\date{\today}

\begin{document}

\maketitle

\section{Introducción} \label{S1}
\subsection*{Contexto}
En el proyecto se modela una compañía de seguros de daños donde los titulares de póliza o asegurados realizan reclamaciones
de acuerdo a procesos de Poisson independientes con una tasa en comun $\lambda$ y el costo de cada reclamación tiene distribución
$F$. Nuevos clientes se unen a la compañia acorde a un proceso de Poisson con un ratio $v$. Cada cliente permanece en la compañía
por un tiempo exponencial con tasa $\mu$ antes de darse de baja. Cada asegurado paga a la compañía de seguros una cantidad $c$ por 
unidad de tiempo. 

\subsection*{Objetivos}
Dada una condición inicial de $n_0$ clientes y un capital inicial $a_0 \geq 0$, se busca estimar que a probabilidad de que el 
capital de la compañía nunca sea negativo en el intervalo de tiempo $[0, t]$."

\subsection*{Variables que describen el problema}

Para simular el sistema anterior, definimos las variables y eventos de la siguiente manera:
\\
Variables: 
\begin{itemize}
    \item Variable de tiempo: $t$.
    \item Variable de estado del sistema $(n,a)$, donde $n$ es el numero de asegurados y $a$ es el capital actual de la compañía.
    \item Tasa de llegada de nuevos clientes: $v$
    \item Tasa de abandono por cliente: $\mu$
    \item Tasa de reclamaciones por clientes: $\lambda$
\end{itemize}
Eventos:
\begin{itemize}
    \item LLegada de un nuevo asegurado(titulares)
    \item Pérdida de un asegurado
    \item Reclamación de un asegurado
\end{itemize}
La lista de eventos consiste en un único valor: el tiempo en el que ocurrirá el próximo evento.\\
Notación: $EL = t_E$
Si el estado actual es $(n,a)$ en tiempo $t$, el próximo evento ocurirrá en $t + X$, donde $X$ es una variable
aleatoria exponencial con tasa:
Tasa Total $$ = v + n\mu + n\lambda$$
Aún así, no importa cuando ocurra el próximo evento, este ocurrirá con una probabilidad:
\begin{itemize}
    \item Nuevo asegurado: $\frac{v}{v+n\mu+n\lambda}$
    \item Pérdida de asegurado: $\frac{n\mu}{v+n\mu+n\lambda}$
    \item Reclamación: $\frac{n\lambda}{v+n\mu+n\lambda}$
\end{itemize}
Tras determinar cuándo ocurre el proximo evento se genera un número aleatorio para identificar
cuál de los tres eventos ocurrió. Luego se actualiza el estado del sistema $(n,a)$ en función del evento seleccionado.

Para un estado $(n,a)$:
\begin{itemize}
    \item $X$: Variable aleatorio exponencial con tasa $v+n\mu+n\lambda$(tiempo hasta el próximo evento).
    \item $J$: Variable aleatoria que representa el tipo de evento.
        \begin{equation}
            J = 
            \begin{cases}
                1 & \text{Nuevo asegurado}, \quad \text{con probabilidad } \dfrac{\nu}{\nu + n\mu + n\lambda}, \\
                2 & \text{Pérdida de asegurado}, \quad \text{con probabilidad } \dfrac{n\mu}{\nu + n\mu + n\lambda}, \\
                3 & \text{Reclamación}, \quad \text{con probabilidad } \dfrac{n\lambda}{\nu + n\mu + n\lambda}
            \end{cases}
        \end{equation}
    \item $Y$: Variable aleatoria con distribución $F$(costo de la reclamación)
    \item $I$: Indicador del éxito financiero:
        \begin{equation}
            I =
            \begin{cases}
                1 & \text{, si el capital es no negativo}, \\
                2 & \text{, en caso contrario}
            \end{cases}
        \end{equation}
\end{itemize}

\section{Detalles de Implementación} \label{S2}

\subsection*{Inicialización}
Para simular el sistema, inicializamos las variables de la siguiente manera:

\begin{enumerate}
    \item \textbf{Inicialización inicial}:
    \begin{align*}
        t &= 0, \\
        a &= a_0, \\
        n &= n_0
    \end{align*}
    
    \item Generar $X \sim \text{Exponencial}(\nu + n\mu + n\lambda)$ e inicializar:
    \begin{align*}
        t_E &= X
    \end{align*}
\end{enumerate}

\subsection*{Actualización}
Para actualizar el sistema, avanzamos al siguiente evento, verificando primero si nos lleva más allá del tiempo $T$:

\begin{enumerate}
    \item \textbf{Caso 1}: Si $t_E > T$:
    \begin{itemize}
        \item Asignar $I = 1$ y finalizar esta ejecución.
    \end{itemize}
    
    \item \textbf{Caso 2}: Si $t_E \leq T$:
    \begin{enumerate}
        \item Reiniciar:
        \begin{align*}
            a &= a + n \cdot c \cdot (t_E - t)  \\
            t &= t_E \quad 
        \end{align*}
        
        \item Generar $J$:
        \begin{align*}
            J &= 
            \begin{cases}
                1: & n = n + 1 \\
                2: & n = n - 1  \\
                3: & 
                \begin{aligned}
                    &\text{Generar } Y \sim F. \\
                    &\text{Si } Y > a \text{, asignar } I = 0 \text{ y terminar.} \\
                    &\text{En otro caso, } a = a - Y \quad \text{(Pagar reclamación)}
                \end{aligned}
            \end{cases}
        \end{align*}
        
        \item Generar $X \sim \text{Exponencial}(\nu + n\mu + n\lambda)$ y actualizar:
        \begin{align*}
            t_E &= t + X
        \end{align*}
    \end{enumerate}
\end{enumerate}

El paso de actualización se repite continuamente hasta que se completa una ejecución.

\section{Resultados y Experimentos} \label{S3}
\subsection*{Hallazgos de la simulación}
Resultados principales obtenidos.

\subsection*{Interpretación de los resultados}
Análisis de lo que significan los resultados.

\subsection*{Hipótesis extraídas}
\begin{itemize}
    \item Hipótesis 1
    \item Hipótesis 2
\end{itemize}

\subsection*{Experimentos de validación}
Descripción de los experimentos realizados para validar las hipótesis.

\subsection*{Análisis estadístico}
\begin{itemize}
    \item Variables de interés
    \item Métodos estadísticos utilizados
\end{itemize}

\subsection*{Análisis de parada}
Criterios utilizados para determinar cuándo detener la simulación.

\section{Modelo Matemático} \label{S4}
\subsection*{Descripción del modelo}
Formulación matemática del modelo, incluyendo modelos probabilísticos si aplica.

\subsection*{Supuestos y restricciones}
\begin{itemize}
    \item Supuesto 1
    \item Supuesto 2
    \item Restricción 1
\end{itemize}

\subsection*{Comparación con resultados experimentales}
Análisis de cómo los resultados del modelo se comparan con los experimentales.

\section{Conclusiones} \label{S5}
\begin{itemize}
    \item Conclusión principal
    \item Logros alcanzados
    \item Posibles extensiones o trabajo futuro
\end{itemize}

\end{document}